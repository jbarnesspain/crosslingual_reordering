\documentclass[a4paper,11pt,twocolumn,twoside]{article}
\usepackage[dvips]{graphicx}
\usepackage{sepln}
\usepackage{fullname_esp}
\usepackage[spanish,es-nosectiondot, es-tabla, es-noindentfirst]{babel}
\usepackage[utf8]{inputenc}
\input epsf

\usepackage[colorinlistoftodos]{todonotes}
\usepackage{multirow}
\usepackage{xspace}
\usepackage{url}
\usepackage{tabularx}
\usepackage{microtype}
\usepackage{graphicx}
\usepackage{paralist}
\usepackage{subfig}
\usepackage{booktabs}
\usepackage{amsmath}
\usepackage{amssymb}

\definecolor{lightgreen}{RGB}{200,255,200}
\definecolor{lightblue}{RGB}{200,200,255}
\definecolor{lightred}{RGB}{255,200,200}

\newcommand{\R}{\mathbb{R}}
\DeclareMathOperator*{\argmin}{arg\,min}
\newcommand{\mt}{\textsc{Mt}\xspace}
\newcommand{\blse}{\textsc{Blse}\xspace}
\newcommand{\barista}{\textsc{Barista}\xspace}
\newcommand{\vecmap}{\textsc{VecMap}\xspace}
\newcommand{\muse}{\textsc{Muse}\xspace}

\newcommand{\splitsent}{\textsc{Split}\xspace}
\newcommand{\sent}{\textsc{Sent}\xspace}
\newcommand{\cononly}{\textsc{Context-only}\xspace}
\newcommand{\asponly}{\textsc{Aspect-only}\xspace}

\newcommand{\cononlytable}{\textsc{\shortstack[1]{Context-\\only}}}
\newcommand{\asponlytable}{\textsc{\shortstack[1]{Aspect-\\only}}}


\newcommand{\rt}[1]{\rotatebox{90}{#1}}
\newcommand{\rrt}[1]{\rotatebox{45}{#1}}

\newcommand{\ie}{\textit{i.\,e.}\xspace}
\newcommand{\eg}{\textit{e.\,g.}\xspace}
\newcommand{\lcon}{\textrm{con}_\ell}
\newcommand{\rcon}{\textrm{con}_r}

\newcommand{\F}{$\text{F}_1$\xspace}

\setlength\titlebox{4in} %esto por defecto

\title{On the Effect of Word Order on Cross-lingual Sentiment Analysis}

\author {\textbf{Álex Ramírez Atrio$^1$}, \textbf{Toni Badia$^{1}$} \textbf{Jeremy Barnes$^{2}$}\\
$^1$Universitat Pompeu Fabra\\
$^2$Universitetet i Oslo\\
Información de contacto\\
}

\seplntranstitle{Sobre el Efecto del Orden de Palabras en el Análisis de Sentimiento Cross-lingüe}

\seplnclave{análisis de sentimiento, cross-lingüe, orden de palabras, análisis}

\seplnresumen{Resumen del artículo en castellano con una sangría a izquierda y
derecha de 1 cm, justificado por ambos lados, con tamaño de fuente
11.}


\seplnkey{sentiment analysis, cross-lingual, word order, analysis}

\seplnabstract{Resumen del artículo en inglés con una sangría a izquierda y
derecha de 1 cm, justificado por ambos lados, con tamaño de fuente
11.}

\firstpageno{1}


\begin{document}

% la siguiente instrucción sólo se debe usar si el abstract sobrescribe el texto
% la longitud variará según se necesite

%\setlength\titlebox{20cm} % se aumenta el tamaño del espacio reservado para datos de título


\label{firstpage} \maketitle

%\begin{abstract}
%Resumen del artículo con una sangría a izquierda y derecha de 0.32
%cm, justificado por ambos lados, con tamaño de fuente 11.
%
%\end{abstract}

\section{Introduction}

\begin{itemize}
\item intro to task: Why is sentiment analysis cool/useful/difficult?
\item motivation for cross-lingual approaches: We often have no annotated data for Language X, especially for specific domains.
\item why it's interesting to use no MT: under-resourced languages, MT requires too much parallel data
\item what problem that might introduce
\end{itemize}

\section{Related Work}

\begin{itemize}

\item Cross-lingual Sentiment Approaches that are relevant here: under-resourced langs
\item Bilingual Word Embeddings: Artetxe and why we use these: SOTA and low-resource
\item Word order in sentiment:
\item Reordering for machine translation
\end{itemize}

\subsection{Cross-lingual Sentiment Analysis}

\cite{Mohammad2015b}

\subsection{Bilingual Word Embeddings}

\subsection{Reordering for Machine Translation}
Crego etc.

\section{Methodology}

\subsection{Models}

\begin{itemize}
\item LSTM, CNN, SVM
\item Differences between how models handle word order
\end{itemize}

\subsection{Corpora and Datasets}

\begin{itemize}
\item OpeNER, Multibooked
\item Europarl, Tatoeba
\item motivation for using these resources
\end{itemize}

\subsection{Experimental Setup}

\begin{itemize}
\item Test all models on two cross-lingual setups (en-es, en-ca)
\item Compare: No reordering, Random Reordering, ONE, No lexicon, Only lexicon
\item What are the competing hypotheses for each of these setups?
\end{itemize}

\section{Results}

\begin{table*}[t]
\newcommand{\sep}{\cmidrule(r){3-5}\cmidrule(r){6-8}}
\newcommand{\sepp}{\cmidrule(r){3-3}\cmidrule(r){4-4}\cmidrule(r){5-5}\cmidrule(r){6-6}\cmidrule(r){7-7}\cmidrule(r){8-8}}

\definecolor{green}{RGB}{150,255,150}
\definecolor{blue}{RGB}{150,150,255}

\newcommand{\bestproj}[1]{{\setlength{\fboxsep}{0pt}\colorbox{lightblue}{\textit{#1}}}}
\newcommand{\bestoverall}[1]{{\setlength{\fboxsep}{0pt}\colorbox{lightgreen}{\textbf{#1}}}}

\setlength\tabcolsep{10pt}
\renewcommand*{\arraystretch}{0.8}
\centering\small
\begin{tabular}{llcccccc}
\toprule
& & \multicolumn{3}{c}{Binary} & \multicolumn{3}{c}{4-class} \\
\sep
\multirow{6}{*}{\rt{EN-ES}}
& Original 	&  &  &  &  &  &  \\ 
& Reordered  &  &  &  &  &  &  \\ 
& N-ADJ  &  &  &  &  &  & \\ 
& Random  &  &  &  &  &  & \\ 
& Only Lexicon  &  &  &  &  &  & \\ 
& No Lexicon  &  &  &  &  &  & \\ 
\sepp
\multirow{6}{*}{\rt{EN-CA}} &
  Original &  &  &  &  &  & \\ 
& Reordered  &  &  &  &  &  & \\ 
& N-ADJ &   &  &  &  &  & \\ 
& Random &  &  &  &  &  & \\ 
& Only Lexicon  &  &  &  &  &  & \\ 
& No Lexicon &  &  &  &  &  & \\ 
\bottomrule
\end{tabular}
\caption{Macro \F results for all corpora and techniques. We denote
  the best performing projection-based
  method per column with a \bestproj{blue box} and the best overall method
  per column with a
  \bestoverall{green box}.}
\label{results:all}
\end{table*}

\section{Analysis}

\todo{It would be nice to show that the noise introduced by bilingual embeddings leads to LSTMs not being able to pick up on word order in the target language. We could train monolingual models for Spanish and Catalan and use the random reordering to see the difference.}

\todo{It would also be interesting to look at particular examples of errors that each model suffers. Are they different in each model? Is there any pattern?}

\section{Conclusion and Future Work}


\section*{Agradecimientos}

We would like to thank ...


\bibliographystyle{fullname_esp}
\bibliography{lit}

\appendix
\section{Appendix 1: Reordering Rules} 



\section{Appendix 2: } 

\end{document}
