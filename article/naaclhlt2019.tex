%
% File naacl2019.tex
%
%% Based on the style files for ACL 2018 and NAACL 2018, which were
%% Based on the style files for ACL-2015, with some improvements
%%  taken from the NAACL-2016 style
%% Based on the style files for ACL-2014, which were, in turn,
%% based on ACL-2013, ACL-2012, ACL-2011, ACL-2010, ACL-IJCNLP-2009,
%% EACL-2009, IJCNLP-2008...
%% Based on the style files for EACL 2006 by 
%%e.agirre@ehu.es or Sergi.Balari@uab.es
%% and that of ACL 08 by Joakim Nivre and Noah Smith

\documentclass[11pt,a4paper]{article}
\usepackage[hyperref]{naaclhlt2019}
\usepackage{times}
\usepackage{latexsym}
\usepackage[utf8]{inputenc}

\usepackage{multirow}
\usepackage{xspace}
\usepackage{url}
\usepackage{tabularx}
\usepackage{microtype}
\usepackage{graphicx}
\usepackage{paralist}
\usepackage{subfig}
\usepackage{booktabs}
\usepackage{amsmath}
\usepackage{amssymb}
\usepackage{xstring}
\usepackage{scalefnt}
\usepackage{bm}
\usepackage{paralist}
%\usepackage{microtype} % microtype breaks non-ascii use in bib
\usepackage{tablefootnote}
\usepackage{subfig}
\usepackage{capt-of}
\usepackage{latexsym}
\usepackage{paralist}
\usepackage{calc}
\usepackage{booktabs}
\usepackage{url}
\usepackage{scrextend}

\newcommand\footnoteref[1]{\protected@xdef\@thefnmark{\ref{#1}}\@footnotemark}
\newcommand{\aspectbox}[1]{{\setlength{\fboxsep}{1pt}\colorbox{lightgreen}{#1}}}
\newcommand{\posbox}[1]{{\setlength{\fboxsep}{1pt}\colorbox{lightblue}{#1}}}
\newcommand{\negbox}[1]{{\setlength{\fboxsep}{1pt}\colorbox{lightred}{#1}}}

\definecolor{lightgreen}{RGB}{200,255,200}
\definecolor{lightblue}{RGB}{200,200,255}
\definecolor{lightred}{RGB}{255,200,200}
\definecolor{lighterred}{RGB}{255,220,220}

\newcommand{\R}{\mathbb{R}}
\DeclareMathOperator*{\argmin}{arg\,min}

\newcommand{\original}{\textsc{Original}\xspace}
\newcommand{\mtreordered}{\textsc{Reordered}\xspace}
\newcommand{\nadj}{\textsc{Noun-Adj.}\xspace}
\newcommand{\random}{\textsc{Random}\xspace}
\newcommand{\onlylex}{\textsc{Only-Lexicon}\xspace}
\newcommand{\nolex}{\textsc{No-Lexicon}\xspace}

\newcommand{\bilstm}{\textsc{BiLstm}\xspace}
\newcommand{\cnn}{\textsc{Cnn}\xspace}
\newcommand{\svm}{\textsc{Svm}\xspace}
\newcommand{\bilstms}{\textsc{BiLstms}\xspace}
\newcommand{\cnns}{\textsc{Cnns}\xspace}
\newcommand{\Rnns}{\textsc{Rnns}\xspace}
\newcommand{\svms}{\textsc{Svms}\xspace}


\newcommand{\rt}[1]{\rotatebox{90}{#1}}
\newcommand{\rrt}[1]{\rotatebox{45}{#1}}

\newcommand{\ie}{\textit{i.\,e.}\xspace}
\newcommand{\eg}{\textit{e.\,g.}\xspace}
\newcommand{\F}{$\text{F}_1$\xspace}


%jer: automatically turned off when aclfinalcopy is turned on
\usepackage{annotates}
\newcommand{\todo}[1]{\comment{Todo}{#1}}
\newcommand{\jer}[1]{\comment{Jeremy}{#1}}
\newcommand{\toni}[1]{\comment{Toni}{#1}}
\newcommand{\alex}[1]{\comment{Alex}{#1}}

\newcommand{\jeranswer}[1]{\answer{Jeremy}{#1}}
\newcommand{\alexanswer}[1]{\answer{Alex}{#1}}
\newcommand{\tonianswer}[1]{\answer{Toni}{#1}}

\newcommand{\jerin}[1]{\inline{Jeremy}{#1}}
\newcommand{\toniin}[1]{\inline{Toni}{#1}}
\newcommand{\alexin}[1]{\inline{Àlex}{#1}}



\usepackage{url}

%\aclfinalcopy % Uncomment this line for the final submission
\def\aclpaperid{345} %  Enter the acl Paper ID here

%\setlength\titlebox{5cm}
% You can expand the titlebox if you need extra space
% to show all the authors. Please do not make the titlebox
% smaller than 5cm (the original size); we will check this
% in the camera-ready version and ask you to change it back.

\newcommand\BibTeX{B{\sc ib}\TeX}

\title{On the Effect of Word Order on Cross-lingual Sentiment Analysis}

\author {\textbf{Àlex R. Atrio$^1$}, \textbf{Toni Badia$^{1}$}, \textbf{Jeremy Barnes$^{2}$}\\[5pt]
$^1$Universitat Pompeu Fabra\\
{\tt alexratrio@gmail.com} \\[1pt]
{\tt toni.badia@upf.edu} \\[5pt]
$^2$University of Oslo\\
{\tt jeremycb@ifi.uio.no}
}

%\author{First Author \\
%  Affiliation / Address line 1 \\
%  Affiliation / Address line 2 \\
%  Affiliation / Address line 3 \\
%  {\tt email@domain} \\\And
%  Second Author \\
%  Affiliation / Address line 1 \\
%  Affiliation / Address line 2 \\
%  Affiliation / Address line 3 \\
%  {\tt email@domain} \\}

\date{}
\begin{document}
\maketitle
\begin{abstract}
Current state-of-the-art models for sentiment analysis make use of word order
either explicitly by pretraining on language modeling or implicitly
by using RNNs or \cnns.
This is a problem for cross-lingual models that
use bilingual embeddings as features, as the difference
in word order between source and target languages is
not resolved. In this work, we explore reordering
as a preprocessing step for sentence-level cross-lingual sentiment
classification with two language combinations
(English-Spanish, English-Catalan). We find that reordering helps both \bilstms and \cnns, and that a simple reordering
approach outperforms a more complex MT-inspired reordering
scheme.


\end{abstract}


\section{Introduction}


Cross-lingual Sentiment Analysis (CLSA) exploits resources, \eg labeled data of a high-resource language, to train a sentiment classifier in order for low-resource languages. This is relevant when a target language lacks plentiful labeled data, particularly for specific domains. Machine Translation (MT) is often used to bridge the gap between languages \cite{Banea2008,Balahur2014d}, but requires parallel annotated data, which may be difficult to find for some low-resource languages. Approaches that use bilingual distributional representations, in contrast, have proven competitive while requiring less parallel data \cite{Chen2016,Barnes2018b}.

Recently, sentiment classifiers pre-trained on a language modeling task have lead to state-of-the-art results \cite{Peters2018,Howard2018,Devlin2018}. This suggests that sentiment analysis benefits from learning word order and fine-grained relationships between tokens, which can be gleaned from unlabeled data. These approaches, however, have only been applied in a monolingual setting and it is not clear how the difference in word orders would affect them in a cross-lingual setup.

In this work, we perform an analysis of the effect of word order on cross-lingual sentiment classifiers that use bilingual embeddings as features. We show that these models benefit from reordering the target-language test data so that it resembles the source-language word order.


\section{Related Work}

\paragraph{Cross-lingual Sentiment Analysis: }

Although most approaches to cross-lingual sentiment analysis rely on machine translation \cite{Banea2008,Balahur2014d,Klinger2015},
there are several approaches that instead rely on bilingual representations. 
\newcite{Prettenhofer2011b} adopt Structural Correspondence Learning to the
bilingual setting. Bilingual word embeddings \cite{Kocisky2014,Chandar2014,Luong2015} now provide
a simple framework for cross-lingual approaches to natural language processing. Current state-of-the-art approaches \cite{Artetxe2017,Artetxe2018,Lample2017} first separately create monolingual embeddings using large, unlabeled corpora in the source and target languages. They then learn an orthogonal projection from the source to the target space in a supervised manner using a bilingual lexicon. 

Bilingual word embeddings are now used as features for state-of-the-art document-level \cite{Chen2016}, sentence-level \cite{Barnes2018b}, and targeted \cite{Hangya2018} cross-lingual sentiment analysis approaches.


\paragraph{Word Order in Sentiment Analysis: }

Pre-training sentiment classifiers with a language-modeling task represents a successful transfer learning method. \newcite{Peters2018} learn
to create contextualized embeddings by training a character-level convolutional
network to predict the next word in a a sequence. Similarly, \newcite{Howard2018} introduce techniques that improve the fine-tuning of the base language-model. Finally, 
\newcite{Devlin2018} introduce a self-attention network, and adjust the language
modeling task to a cloze task, where they predict missing words in a sentence, rather than the next word given a sequence. They then fine-tune their models on downstream tasks. These models that explicitly learn word order have led to state-of-the-art results on sentiment tasks.

\paragraph{Word Reordering: }

In a survey on word reordering in Statistical Machine Translation, \newcite{Bisazza2016} mention that because of some problems we may find in MT, it is often best to preprocess the source language by reordering it and then carrying out the translation. Reordering rules can be determined manually \cite{Crego2006,Crego2006b} or with data-driven approaches \cite{Neubig2012,Nakagawa2015}. Their application can be deterministic or non-deterministic. Some hybrid techniques exist as well, where long-range transformations are deterministic and the rest non-deterministic.

\section{Methodology}

\subsection{Corpora and Datasets}

\begin{table}[tb]
\centering%\small
\begin{tabular}{lrrrr}
\toprule
    & & \multicolumn{1}{c}{EN} & \multicolumn{1}{c}{ES} & \multicolumn{1}{c}{CA} \\
\cmidrule(rl){2-2}\cmidrule(l){3-3}\cmidrule(l){4-4}\cmidrule(l){5-5}
 \multirow{2}{*}{\rt{Binary}}
 &$+$   & 1258 & 1216 & 682     \\
 &$-$   & 473 & 256 & 467   \\
\cmidrule(rl){2-2}\cmidrule(l){3-3}\cmidrule(l){4-4}\cmidrule(l){5-5}
 \multirow{4}{*}{\rt{4-class}}
 &$++$   & 379 & 370  & 256  \\
 &$+$    & 879 & 846  & 426   \\
 &$-$    & 399 & 218  & 409    \\
 &$--$   &  74 & 38   & 58     \\
 \cmidrule(rl){2-2}\cmidrule(l){3-3}\cmidrule(l){4-4}\cmidrule(l){5-5}
 &\textit{Total}     & 1731  & 1472     & 1149       \\
\bottomrule
\end{tabular}
\caption{Statistics for the OpeNER English (EN) and Spanish (ES) 
as well as the MultiBooked Catalan (CA) sentence-level datasets \citep{Agerri2013,Barnes2018a}.}
\label{datasetstats}
\end{table}

At document-level, bag-of-words models are often expressive enough to give good results without relying on word order. But because we are interested in word-order effects in cross-lingual sentiment analysis, we require datasets that are annotated at a fine-grained level, \ie sentence- or aspect-level.

For this reason, we use the English and Spanish OpeNER corpora of hotel reviews \cite{Agerri2013} as well as the Catalan MultiBooked Dataset \cite{Barnes2018a}. Each sentence is annotated for four classes of sentiment\footnote{Strong positive, positive, negative, and strong negative.}. We use the English subset for training our classifiers and the Spanish and Catalan for testing the effects of word order on the target languages. Although these datasets are relatively small, they are all annotated similarly and
are in domain, which avoids problems with mapping labels or domain shifts.

\subsection{Bilingual Word Embeddings}

VecMap \cite{Artetxe2016,Artetxe2017} creates bilingual embeddings by learning an orthogonal projection between two precomputed monolingual vector spaces and requires only a small bilingual dictionary. We use the publicly available GoogleNews vectors for the English\footnote{Available at \url{https://code.google.com/archive/p/word2vec/}}, and for Spanish and Catalan we create skip-gram embeddings with 300 dimensions trained on Wikipedia data. The bilingual dictionaries are translated sentiment lexicons with 5700 pairs for English -- Spanish (5271 for English -- Catalan).



\subsection{Experimental Setup}

\begin{table*}[]
\centering\small
\newcommand{\lex}[1]{{\setlength{\fboxsep}{1pt}\colorbox{lightred}{\textbf{#1}}}}
\newcommand{\nonlex}[1]{{\setlength{\fboxsep}{1pt}\colorbox{lighterred}{\textit{#1}}}}
\begin{tabular}{ll}
\toprule
\emph{\original} & Único punto \lex{negativo} el \lex{ruido} que las ventanas de madera tan típicas de la zona \nonlex{no} \nonlex{consiguen} \lex{aislar} \\[3pt]
\emph{\mtreordered} & Único \lex{negativo} punto el \lex{ruido} que las ventanas de tan típicas madera de la \nonlex{no} zona \nonlex{consiguen} \lex{aislar} \\[3pt]
\emph{\nadj} & Único \lex{negativo} punto el \lex{ruido} que las ventanas de madera tan típicas de la zona \nonlex{no} \nonlex{consiguen} \lex{aislar} \\[3pt]
\emph{\random} & aislar madera \nonlex{consiguen} típicas de el de zona las ventanas punto \lex{negativo} Único la \nonlex{no} \lex{ruido} tan que\\[3pt]
\emph{\onlylex} & UNK \lex{negativo} UNK UNK \lex{ruido} UNK UNK UNK UNK UNK UNK UNK UNK UNK UNK UNK  \lex{aislar}\\[3pt]
\emph{\nolex} & Único punto UNK el UNK que las ventanas de madera tan típicas de la zona \nonlex{no} \nonlex{consiguen} UNK \\[3pt]
\cmidrule(lr){0-1}
\emph{Translation} & The only \lex{negative} point the \lex{noise} that the typical wooden windows in the area \nonlex{don't manage} to \lex{block} \\[3pt]
\bottomrule
\end{tabular}
\caption{An example of a negative Spanish sentence (\original) with the five reordering
transformations applied, as well as its English translation. The \lex{bold tokens} are words found in the sentiment lexicon, and the \nonlex{italic words} are words that convey sentiment in this instance, but are not in the lexicon.}
\label{example}

\end{table*}

In order to test whether a sentiment classifier trained on bilingual embeddings is sensitive to word order, we test classifiers on six versions of the target-language sentiment data, which we describe in the following section. An example of these six versions is shown in Table \ref{example}.

\paragraph{\original: }The model is tested on the original data with no changes in word order.

\paragraph{\mtreordered: }A competing hypothesis is that a full pre-reordering of the target-side sequences will be more familiar to the sentiment classifier trained on English and therefore lead to better results. We implement a version of linguistically motivated rewrite rules \cite{Crego2006,Crego2006b}, which are then applied to the target-language test data before testing.

\paragraph{\nadj: }Given that adjectives are important for sentiment analysis, we hypothesize that adjusting the order of nouns and adjectives should be beneficial if the classifier is learning source-language word order. Therefore, we implement a simple reordering where Spanish and Catalan adjectives are placed in front of the noun they modify, rather than following it.

\paragraph{\random: }We randomly permutate the order of the target-language sentences. If the sentiment classification models take the target language word order into consideration, this should lead to poor results.

\paragraph{\onlylex and \nolex: }Finally, we provide two baselines for clarification. The \onlylex experiment removes all words which do
not appear in the Hu \& Liu sentiment lexicon \cite{HuandLiu2004}. If our systems
take word order into account, they should be affected negatively by this, as the
resulting sentence does not resemble the normal word order. If, however, the models
are simply relying on keywords, this will have little effect.

For the \emph{\nolex} experiment, we remove all of the words
in a phrase which are found in the sentiment lexicon. If the models are simply attending to
sentiment keywords, this approach should lead to the worst performance. 

We perform additional experiments with English in a monolingual setup and using a MT-based approach. For these two setups, we only test the \random reordering, \onlylex, and \nolex approaches. Note that the monolingual setup is not comparable, as we have a smaller test set.

\subsection{Models}

To test our hypotheses, we compare three different classifiers: a Support Vector Machine (SVM) with Bag-of-Embeddings, a Convolutional Neural Network (\cnn) \cite{Santos2014,Severyn2015}, and a Bidirectional Long Short Term Memory Network (\bilstm) \cite{Luong2015}. Each of these classifiers theoretically have an increasing reliance on word order. Although the SVM does not take into account word order at all, it is a strong baseline for sentiment analysis \cite{Kiritchenko2014c}. The \cnn considers only local word order, while the \bilstm relies on both local and long-distance dependencies.

We implement a single-layered \bilstm classifier with a 100-dimensional hidden layer, which passes the concatenation of the two final hidden states to a softmax layer for classification. The \cnn has convolutional filters $\in \{3,4,5\}$ and a max-pooling layer of length $2$. Both models are initialized with the pretrained bilingual embeddings, use dropout of $0.3$ for regularization, and are trained for 100 epochs with a batch size of 32 using Adam as an optimizer. We choose the parameter for training epochs on the source-language development set and test this model on the target-language data. Finally, we implement a baseline bag-of-embeddings SVM, which is implemented in sklearn, while both neural models are implemented in Keras.

\section{Results}



\begin{table}
\newcommand{\sep}{\cmidrule(r){4-6}\cmidrule(r){7-9}}
\newcommand{\sepp}{\cmidrule(r){4-4}\cmidrule(r){5-5}\cmidrule(r){6-6}\cmidrule(r){7-7}\cmidrule(r){8-8}\cmidrule(r){9-9}}

\definecolor{green}{RGB}{150,255,150}
\definecolor{blue}{RGB}{150,150,255}

\newcommand{\bestproj}[1]{{\setlength{\fboxsep}{0pt}\colorbox{lightblue}{\textbf{#1}}}}
\newcommand{\bestmono}[1]{{\setlength{\fboxsep}{0pt}\colorbox{lightgreen}{\textbf{#1}}}}
\newcommand{\bestmt}[1]{{\setlength{\fboxsep}{0pt}\colorbox{pink}{\textbf{#1}}}}

\setlength\tabcolsep{4pt}
\renewcommand*{\arraystretch}{0.5}
\centering\tiny
%\centering
\begin{tabular}{lllcccccccccccc}
\toprule
&& & \multicolumn{3}{c}{Binary} & \multicolumn{3}{c}{4-class} \\
\sep
\multirow{16}{*}{\rt{Bilingual Word Embeddings}} 
	    &&& \bilstm & \cnn & SVM & \bilstm & \cnn & SVM \\
	    \cmidrule(r){4-4}\cmidrule(r){5-5}\cmidrule(r){6-6}\cmidrule(r){7-7}\cmidrule(r){8-8}\cmidrule(r){9-9}
	& \multirow{6}{*}{\rt{EN-ES}}
		& \original 	 & 61.6 & 59.8 & 66.6 & 35.9 & 36.3 & 34.9 \\ 
		&& \mtreordered  & \bestproj{61.9} & 59.9 & 66.6 & 35.7 & \bestproj{36.5} & 34.9 \\ 
		&& N-ADJ  & 61.6 & \bestproj{60.2} & 66.6 & \bestproj{36.3} & \bestproj{36.5} & 34.9 \\ 
		&& \random  & 60.4 & 58.2 & 66.6 & 34.2 & 35.3 & 34.9 \\ 
		&& \onlylex  & 59.1 & 26.2 & 53.0 & 29.8 & 16.4 & 30.7 \\ 
		&& \nolex  & 57.9 & 55.7 & 63.4 & 33.1 & 33.7 & 33.3 \\ 
	\sepp
	& \multirow{6}{*}{\rt{EN-CA}}
  		& \original & 68.2 & 65.7 & 68.2 & 31.9 & \bestproj{39.3} & 33.2 \\ 
		&& \mtreordered  & 68.4 & 65.5 & 68.2 & \bestproj{32.3} & 38.1 & 33.2 \\ 
		&& N-ADJ &  \bestproj{68.5} & \bestproj{65.8} & 68.2 & 31.8 & 38.9 & 33.2 \\ 
		&& \random & 66.9 & 64.4 & 68.2 & 31.7 & 36.4 & 33.2 \\ 
		&& \onlylex  & 55.1 & 41.3 & 39.1 & 30.3 & 27.4 & 23.8 \\ 
		&& \nolex & 64.8 & 63.0 & 63.1 & 30.4 & 35.9 & 31.2 \\ 



\\
\hline \\

\multirow{9}{*}{\rt{Machine Translation}}
	& \multirow{4}{*}{\rt{EN (ES)}}
  		& \original & \bestmt{71.1} & \bestmt{67.4} & 69.6 & 47.1 & \bestmt{43.9} & 45.1 \\ 
		&& \random & 71.0 & 65.9 & 69.6 & \bestmt{48.0} & 41.7 & 45.1 \\ 
		&& \onlylex  & 61.8 & 47.3 & 54.5 & 32.2 & 27.3 & 35.7 \\ 
		&& \nolex & 63.2 & 59.1 & 65.4 & 42.9 & 37.9 & 42.0 \\ 
	\sepp
	& \multirow{4}{*}{\rt{EN (CA)}}
  		& \original & \bestmt{79.0} & \bestmt{77.3} & 74.7 & \bestmt{53.1} & \bestmt{49.2} & 45.5 \\ 
		&& \random & 76.1 & 75.7 & 74.7 & 46.3 & 41.6 & 45.5 \\ 
		&& \onlylex  & 52.1 & 59.6 & 43.9 & 26.1 & 30.1 & 32.7 \\ 
		&& \nolex & 73.0 & 71.5 & 70.8 & 46.7 & 43.1 & 44.4 \\ 
		
\\[3pt]
\hline \\

\multirow{13}{*}{\rt{Monolingual}}
%	& \multirow{4}{*}{\rt{EN}}
%  		& \original &       \bestmono{78.6} & \bestmono{68.0} & 72.2 & \bestmono{53.3} & \bestmono{51.7} & 43.9 \\ 
%		&& \random &        76.8 & 67.2 & 72.2 & 52.8 & 42.6 & 43.9 \\ 
%		&& \onlylex.  & 60.5 & 60.1 & 45.1 & 37.0 & 32.7 & 39.1 \\ 
%		&& \nolex. &    70.6 & 61.1 & 70.7 & 46.5 & 45.2 & 40.3 \\ 
%		\sepp
    & \multirow{4}{*}{\rt{ES}}
  		& \original &    \bestmono{73.9} & \bestmono{66.1} & 49.7 & \bestmono{43.2} & \bestmono{38.0} & 32.1\\
		&& \random &     71.7 & 60.4 & 49.7 & 41.7 & 33.6 & 32.1\\ 
		&& \onlylex  & 47.2 & 46.5 & 45.2 & 26.0 & 25.6 & 27.0\\ 
		&& \nolex &    68.2 & 65.2 & 47.9 & 40.1 & 37.2 & 30.3\\ 
		\sepp
	& \multirow{4}{*}{\rt{CA}}
  		& \original &    \bestmono{77.0} & \bestmono{75.2} & 75.0 & \bestmono{50.3} & \bestmono{46.7} & 46.8\\
		&& \random &     73.1 & 67.9 & 75.0 & 46.5 & 40.3 & 46.8\\ 
		&& \onlylex  & 43.0 & 56.7 & 39.6 & 13.9 & 32.2 & 16.7\\ 
		&& \nolex &    73.6 & 75.4 & 74.8 & 48.3 & 45.9 & 45.8\\ 
		
\\

\bottomrule
\end{tabular}
\caption{Macro \F results for all corpora and techniques. We denote
  the best performing bilingual embedding
  method per column with a \bestproj{blue box}, the best MT method with a \bestmt{pink box}, and the best monolingual method
  with a \bestmono{green box}. We do not denote bag-of-words SVM results, as they are invariant to word order. Monolingual results are not comparable to BWE or MT.}
\label{results:all}
\end{table}

Table \ref{results:all} shows the results of all experiments. Reordering the test data improves the results on seven of the eight experiments. The \mtreordered approach improves the \bilstm results on three of four experiments (two of for for \cnn), while having no effect on the SVM. The simpler \nadj flip performs even better, with a small but consistent improvement over the \original ordering across \bilstm and \cnn experiments\footnote{We do not consider SVM experiments to calculate improvements as they are invariant to word order.}. This confirms that some kind of reordering is beneficial when using neural networks with bilingual embeddings as features.

\random has a larger negative effect on monolingual models (a average decrease of 2.9 ppt for \bilstm and 6.0 for \cnn) than bilingual embedding models or MT-based models (1.1/1.7 and 2.3/3.2, respectively). Additionally, \random has a larger effect on the \cnn than on the \bilstm. This is likely because the \cnn relies on specific combinations of n-grams. If these are not present, the filters are not effective at classification.

Although they are not comparable because they are tested on different datasets, the monolingual models generally perform better than the cross-lingual versions (6.3 -- 12.3 percentage points on binary and 1.7 -- 18.4 on 4-class) with the exception of SVM on Spanish. The machine translation approaches fall in between these two.

The classification models display different trends across the setups. On the monolingual and machine translations setups, the \bilstm is the strongest model, followed by the \cnn and SVM (SVM and \cnn, respectively for machine translation). With bilingual embeddings, however, the SVM outperforms both the \bilstm and \cnn on the Spanish binary setup, while the \cnn is strongest on the multiclass.  This shows that \bilstm displays a different behavior with bilingual embeddings.

The machine translation models perform well, but are surprisingly invariant to word order. They also suffer less from using only features from the sentiment lexicon (\onlylex), which suggests they rely more on these keywords, while ignoring word order effects to a higher degree.

Finally, the \nolex and \onlylex baselines perform poorly. In fact, \onlylex is the worst performing model, which demonstrates that all models are doing more than simply attending to keywords.



\section{Analysis}

Reordering tends to help both the \bilstm and \cnn models with shorter examples (less than 8 tokens long) where adjective order can easily be changed to resemble English word order, such as the examples shown in Table \ref{helpful_examples}. In longer examples (10 + tokens), however, the reordering either introduces too much noise, or has no effect on the final prediction. The current reordering models are therefore more adequate for sentiment tasks that deal with shorter texts, such as aspect- or sentence-level, rather than document-level sentiment analysis.

\begin{table}[]
\centering\scriptsize
\newcommand{\sepp}{\cmidrule(rl){1-1}\cmidrule(rl){2-2}\cmidrule(rl){3-3}}
\begin{tabular}{lll}
\toprule
model & text & prediction \\
\sepp
\emph{\original} & relación calidad precio muy buena & negative \\
\emph{\mtreordered} & relación muy buena calidad precio & positive \\
\textit{translation} & very good quality price relationship & positive \\
\sepp
\emph{\original} & hotel perfecto & negative \\
\emph{\mtreordered} & perfecto hotel & positive \\
\textit{translation} & perfect hotel & positive \\
\sepp
\emph{\original} & el desayuno muy bueno . & negative \\
\emph{\mtreordered} & el muy bueno desayuno . & positive \\
\textit{translation} & the breakfeast (was) very good & positive \\
\sepp
\emph{\original} & gestión nefasta . & positive \\
\emph{\mtreordered} & nefasta gestión . & negative \\
\textit{translation} & terrible management & negative \\
\bottomrule
\end{tabular}
\caption{Examples where reordering improves results over original on binary English-Spanish setup with the BiLSTM classifier.}
\label{helpful_examples}

\end{table}
\section{Conclusion and Future Work}

In this work, we have shown that neural networks that rely on bilingual embeddings as features
are sensitive to word order and benefit from reordering the target language test data. 

Given that state-of-the-art monolingual models explicitly learn word order on language modeling pre-training, it is important to realize that cross-lingual models based on bilingual word embeddings are not currently able to do so. In the future, it may be useful to pretrain bilingual language models in a similar way.

\bibliography{lit}
\bibliographystyle{acl_natbib}



\end{document}
